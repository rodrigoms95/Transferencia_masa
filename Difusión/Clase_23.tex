\documentclass[11pt]{article}
\usepackage[a4paper,width=160mm,top=25mm,bottom=25mm]{geometry}
\usepackage{graphicx}
\usepackage{fancyhdr}
\usepackage[spanish, mexico]{babel}
\usepackage[utf8]{inputenc}
\usepackage{amsmath}
\usepackage{gensymb}
\usepackage{upgreek}
\usepackage{mathtools}
\usepackage{xfrac}
\usepackage{hyperref}
%\usepackage[
%    backend=biber,
%    style=apa,
%  ]{biblatex}
\usepackage{csquotes}
%\addbibresource{bibresource}

\setlength{\headheight}{15pt}
\pagestyle{fancy}
\fancyhf{}
\lhead{Edificios Sustentables}
\chead{Transferencia de calor}
\rhead{Rodrigo Muñoz Sánchez}
\rfoot{\thepage}
\renewcommand{\headrulewidth}{0pt}

\renewcommand{\theenumiii}{\roman{enumiii}}

\title{Clase 25: \\ Ejercicios de Reactores bien mezclados}
\author{Rodrigo Muñoz}
\date{2022}

\graphicspath{Images/}

\begin{document}

\maketitle

\section{Ejercicio 1}

Una ciudad tiene una descarga de aguas residuales de \( 5 \sfrac{\text{m}^3}{\text{s}} \) con una \( DBO_5 \) de \( 300 \sfrac{\text{mg}}{l} \) y una constante de degradación de \( 0.15 \text{día}^{-1} \). Diseña un reactor bien mezclado que permita tratar el agua a \( 100 \left[ 100 \sfrac{\text{mg}}{l} \right] \).

\bigskip \bigskip

Datos del problema:

\[ \chi_I = 300 \left[ \sfrac{\text{mg}}{l} \right] \]

\[ \chi_O = 100 \left[ \sfrac{\text{mg}}{l} \right] \]

\[ Q = 300 \left[ \sfrac{\text{m}^3}{\text{s}} \right] = 5 \cdot 3600 \cdot 24 = 432,000 \left[ \sfrac{\text{m}^3}{\text{día}} \right] \]

\[ k = 0.15 \left[ \text{día}^{-1} \right] \]

Calculamos la eficiencia:

\[ \eta = 1 - \frac{\chi_O}{\chi_I} = 1 - \frac{100}{300} = 0.667 \]

Volumen necesario para lograr la eficiencia:

\[ V = \frac{Q}{k} \left( \frac{\eta}{ 1 - \eta } \right) \]

\[ V = \frac{432,000}{0.15} \left( \frac{\eta}{ 1 - \eta } \right) \]

Si el reactor debe tener una profundia de \( 2 \text{m} \), ¿cuántas hectáreas se requieren de terreno?

\[ z = 2 \left[ \text{m} \right] \]

\[ A = \frac{V}{z} = \frac{5.77 \times 10^6}{2} = 2.89 \times 10^6 \left[ \text{m}^2 \right] \]

Este tipo de sistemas de tratamiento se llaman estanques facultativos, donde se tiene un reactor bien mezclado, con poca profundidad y sin aireación. En realidad el sistema no tiene concentraciones homogéneas, ya que no hay ningún dispositivo de mezclado mecánico, y la sedimentación genera un gradiente vertical de concentración. Como la capa superior está en contacto con el aire, hay un suministro constante de oxígeno, mientras que hacia el fondo solo hay consume de éste por parte de los microorganismos. Esto genera un gradiente de concentración de oxígeno y hace que haya un régimen aerobio en la parte superior del estanque y uno anaerobio en la parte inferior.

\section{Ejercicio 2}

En un frasco se coloca una muestra de aguas residuales que resultan con una \( DBO_5 \) de \( 200 \sfrac{\text{mg}}{l} \). Si suponemos que la constante de degradación es de \( 0.2 \text{día}^{-1} \), ¿cuál será la \( DBO_{10} \), \( DBO_{20} \), y la \( DBO_{\text{última}} \)?

\bigskip \bigskip

Datos del problema:

\[ DBO_5 = 250 \left[ \sfrac{\text{mg}}{l} \right] \]

\[ k = 0.2 \left[ \text{día}^{-1} \right] \]

Ecuación de la \( DBO \):

\[ DBO_t = L_0 \left( 1 - e^{-kt} \right) \]

Con \( t = 5 \) días, y despejando \( L_0 \):

\[ L_0 = \frac{DBO_5}{ 1 - e^{-5k} } \]

\[ L_0 = \frac{250}{ 1 - e^{ -5 \cdot 0.2 } } = 395 \left[ \sfrac{\text{mg}}{l} \right] \]

Con \( t = 10 \) días:

\[ DBO_{10} = L_0 \left( 1 - e^{-10k} \right) = 395.5 \left( 1 - e^{ -10 \cdot 0.2 } \right) \]

\[ \boxed{ DBO_{10} = 342 \left[ \sfrac{\text{mg}}{l} \right] } \]

Con \( t = 20 \) días:

\[ DBO_{20} = L_0 \left( 1 - e^{-20k} \right) = 395.5 \left( 1 - e^{ -20 \cdot 0.2 } \right) \]

\[ \boxed{ DBO_{20} = 388.3 \left[ \sfrac{\text{mg}}{l} \right] } \]

Con \( t = \infty \):

\[ DBO_{\infty} = L_0 \]

\[ \boxed{ DBO_{\text{última}} = 395.5 \left[ \sfrac{\text{mg}}{l} \right] } \]

\section{Ejercicio 3}

Para dos muestras con la misma \( DBO_5 \), ¿qué error se comete en la \( DBO_t \) si las constantes son diferentes?

Establecemos la ecuación de la \( DBO_t \) en función de la \( DBO_5 \):

\[ DBO_t = DBO_5 \left( \frac{ 1 - e^{-kt} }{ 1 - e^{-5k} } \right) \]

El cociente de \( DBO_{ t k_1 } \) y \( DBO_{ t k_2 } \) es:

\[ \frac{DBO_{t \ k_1}}{DBO_{t \ k_2}} = \frac{ 1 - e^{ -k_1 t } }{ 1 - e^{ -5 k_1 } } \frac{ 1 - e^{ -5 k_2 t } }{ 1 - e^{ -k_2 t } } \]

y el error es:

\[ e = 1 - \frac{DBO_{t \ k_1}}{DBO_{t \ k_2}} \]

\[ \boxed{ e = 1 - \frac{ 1 - e^{ -k_1 t } }{ 1 - e^{ -5 k_1 } } \frac{ 1 - e^{ -5 k_2 t } }{ 1 - e^{ -k_2 t } } } \]

que no depende del valor de la \( DBO_5 \).

si \( k_1 = 0.15 \text{día}^{-1} \) y \( k_2 = 0.2 \text{día}^{-1} \) en \( t = 10 \) días:

\[ e = 1 - \frac{ 1 - e^{ -1.5 } }{ 1 - e^{ -0.75 } } \frac{ 1 - e^{ -1 } }{ 1 - e^{ -2 } } \]

\[ e = -7.6 \% \]

En el caso de la \( DBO_\infty \), despejamos \( L_0 \) de la ecuación de la \( DBO_5 \):

\[ DBO_\infty = \frac{DBO_5}{ 1 - e^{-5k} } \]

\[ e = 1 - \frac{DBO_{\infty \ k_1}}{DBO_{\infty \ k_2}} = 1 - \frac{ 1 - e^{-5 k_2} }{ 1 - e^{ -5 k_1 } } \]

y entonces el error sería

\[ e = 1 - \frac{ 1 - e^{-1} }{ 1 - e^{-0.75} } \]

\[ \boxed{ e = -19.8 \% } \]

que es un valor muy alto.

Si la \( DBO_5 = 250 \sfrac{\text{mg}}{l} \):

\[ DBO_{\infty \ k_1} = \frac{250}{ 1 - e^{-0.75} } \left[ \sfrac{\text{mg}}{l} \right] \]

\[ DBO_{\infty \ k_2} = \frac{250}{ 1 - e^{-1} } = 295.5 \left[ \sfrac{\text{mg}}{l} \right] \]


Si calculamos el error con estos datos llegarmos al mismo valor.

Con esto se hace patente la importancia de compara la \( DBO_5 \) entre diferentes muestras solo cuando las constantes de degradación son iguales, o si en conjunto con la \( DBO_5 \) se da el valor de \( k \).

%\nocite{*}
%\printbibliography

\end{document}