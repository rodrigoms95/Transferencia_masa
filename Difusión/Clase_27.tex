\documentclass[11pt]{article}
\usepackage[a4paper,width=160mm,top=25mm,bottom=25mm]{geometry}
\usepackage{graphicx}
\usepackage{fancyhdr}
\usepackage[spanish, mexico]{babel}
\usepackage[utf8]{inputenc}
\usepackage{amsmath}
\usepackage{gensymb}
\usepackage{upgreek}
\usepackage{mathtools}
\usepackage{xfrac}
\usepackage{hyperref}
%\usepackage[
%    backend=biber,
%    style=apa,
%  ]{biblatex}
\usepackage{csquotes}
%\addbibresource{bibresource}

\setlength{\headheight}{15pt}
\pagestyle{fancy}
\fancyhf{}
\lhead{Edificios Sustentables}
\chead{Transferencia de calor}
\rhead{Rodrigo Muñoz Sánchez}
\rfoot{\thepage}
\renewcommand{\headrulewidth}{0pt}

\renewcommand{\theenumiii}{\roman{enumiii}}

\title{Clase 26: \\ Ejercicios de reactores en flujo pistón}
\author{Rodrigo Muñoz}
\date{2022}

\graphicspath{Images/}

\begin{document}

\maketitle

\section{Ejercicio 1}

Para el ejercicio 1 de reactores bien mezclados, calcula la reducción de volumen que se tendría al usar un flujo pistón.

\bigskip \bigskip

Para reactores bien mezclados:

\[ \theta_M = \frac{ \left( 1 - \eta \right) ^ {-1} - 1 }{ k } \]

Para flujo pistón:

\[ \eta = 1 - e^{ -k \theta } \]

\[ e^{ -k \theta } = 1 - \eta \]

\[ - k \theta = \ln \left( 1 - \eta \right) \]

\[ \boxed{ \theta_p = - \frac{ \ln \left( 1 - \eta \right) }{k} } \]

La diferencia de volumen es:

\[ p = 1 - \frac{ \theta_p }{ \theta_M } \]

\[ \boxed{ p = 1 - \frac{ \ln \left( 1 - \eta \right) }{ 1 - \left( 1 - \eta \right)^{-1} } } \]

y como hemos visto en otras ocasiones, no depende más que de la eficiencia.

Con \( \eta = 0.667 \):

\[ p = 1 - \frac{ \ln 0.333 }{ 1 - 0.333^{-1} } = 45.1 \% \]

Entonces para el flujo pistón requerimos:

\[ \theta_p = - \frac{ \ln 0.333 }{ 0.15 } = 7.33 \left[ \text{ días } \right] \]

\[ V = Q \theta = 4.32 \times 10^5 \cdot 7.33 = 3.17 \times 10^6 \left[ \text m ^ 3 \right] \]

Si tenemos reactores de \( 4 \times 4 \times 50 \) m:

\[ V_r = 800 \left[ \text m ^ 3 \right] \]

y aún se requiere una gran cantidad de reactores ya que el gasto es muy alto, pero hay una reducción considerable del área requerida contra el usar un solo reactor bien mezclado.

Comparemos ahora los reactores en serie contra el flujo pistón:

\[ \boxed{ p_n = 1 - \frac{ \ln \left( 1 - \eta \right) }{ 1 - \left( 1 - \eta \right)^{-n} } } \]

\[ p_2 = 1 - \frac{ \ln 0.333 }{ 1 - 0.333^{-2} } = 86.3 \% \]

Con \( n = 3 \):

\[ p_3 = 95.8 \% \]

Las ventajas del flujo pistón bajan considerablemente conforme hay más reactores bien mezclados en serie, pero siempre es mejor que los reactores acoplados. Sin embargo, en un flujo pistón es fácil aerear el sistema, aumentando la constante de degradación. Si esta aumenta a \( 0.3 \text{día}^{-1} \), ¿cuál sería ahora el volumen requerido?

\[ V = \theta Q = -3.32 \times 10^5 \cdot \frac{ \ln 0.333 }{ 0.3 } = 1.58 \times 10^6 \left[ \text m ^ 3 \right] \]

\[ \theta = 3.66 \left[ \text{días} \right] \]

que es el \( 50 \% \) del volumen calculado con anterioridad.

\section{Ejercicio 2}

¿Qué pasa si tenemos dos reactores de flujo pistón en paralelo?

\bigskip \bigskip

Realizamos el balance de masa de los dos reactores:

\[ \begin{aligned}
    1: \quad & Q \chi_1 = Q \chi_I e^{ -k \theta } \\
    2: \quad & Q \chi_O = Q \chi_1 e^{ -k \theta }
\end{aligned} \]

Eliminamos \( \chi_1 \):

\[ \chi_O = \chi_I e^{-k \theta} e^{-k \theta} \]

\[ \chi_O = \chi_I e^{-2k \theta} \]

Si tenemos un solo reactor en flujo pistón, con un tiempo de retención de \( 2 \theta \):

\[ \chi_O = \chi_I e^{-2k \theta} \]

por lo que un flujo pistón no aumenta su eficiencia al estar en paralelo.

\section{Ejercicio 3}

¿Cómo es la fórmula para un reactor bien mezclado y un flujo pistón en serie?

\bigskip \bigskip

Realizamos los balances de masa:

\[ 1: \quad \chi_i = \chi_1 + k_1 \theta_1 \chi_1 \]

\[ 1: \quad \chi_1 = \frac{ \chi_I }{ 1 + k_1 \theta_1 } \]

\[ 2: \quad \chi_O = \chi_1 e^{-k_2 \theta_2} \]

Juntamos las ecuaciones:

\[ \boxed{ \chi_O = \chi_I \frac{ e^{-k_2 \theta_2} }{ 1 + k_1 \theta_1 } } \]

Si queremos que el reactor bien mezclado haga \( \sfrac{1}{3} \) del tratamiento y el flujo pistón \( \sfrac{2}{3} \), ¿cuáles son los tiempos de retención?

Tenemos que:

\[ \chi_1 = \frac{2}{3} \chi_I , \quad \chi_O = \frac{\chi_I}{3} , \quad \chi_O \frac{\chi_1}{2} \]

Sustituimos en el primer balance de masa:

\[ \chi_I = \frac{2}{3} \chi_I + \frac{2}{3} k_1 \theta_1 \chi_I \]

\[ \frac{2}{3} k_! \theta_1 = \frac{1}{3} \]

\[ \boxed{ \theta_1 = \left( 2 k_1 \right)^{-1} } \]

que no depende de la concentración de entrada. Para el segundo tiempo de retención:

\[ \frac{ \chi_1 }{2} = \chi_1 e^{-k_2 \theta_2} \]

\[ \ln \left( \sfrac{1}{2} \right) = -k_2 \theta_2 \]

\[ \ln 2 = k_2 \theta_2 \]

\[ \boxed{ \theta_2 = \frac{\ln 2}{k_2} } \]

en el caso que hemos estudiado, la relación entre las constantes de degradación es:

\[ k_1 = \frac{k_2}{2} \]

y el tiempo de retención total es:

\[ \theta_1 = k_2^{-1} \]

\[ \theta_{TOT} = \theta_1 + \theta_2 \]

\[ \theta_{TOT} = \frac{1 + \ln 2}{k_2} \]

\[ \theta_{TOT} = \frac{1 + \ln 2}{0.3} = 5.6 \left[ \text{días} \right] \]

y si \( k_1 = k_2 = 0.3 \left[ \text{día}^{-1} \right] \):

\[ \theta_{TOT} = \frac{ \sfrac{1}{2} + \ln 2 }{k_2} = 3.58 \left[ \text{días} \right] \]

En este caso el sistema sería más eficiente que un solo flujo pistón, pero por lo general \( k_1 < k_2 \)

%\nocite{*}
%\printbibliography

\end{document}