\documentclass[11pt]{article}
\usepackage[a4paper,width=160mm,top=25mm,bottom=25mm]{geometry}
\usepackage{graphicx}
\usepackage{fancyhdr}
\usepackage[spanish, mexico]{babel}
\usepackage[utf8]{inputenc}
\usepackage{amsmath}
\usepackage{gensymb}
\usepackage{upgreek}
\usepackage{mathtools}
\usepackage{xfrac}
\usepackage{hyperref}
%\usepackage[
%    backend=biber,
%    style=apa,
%  ]{biblatex}
\usepackage{csquotes}
%\addbibresource{bibresource}

\setlength{\headheight}{15pt}
\pagestyle{fancy}
\fancyhf{}
\lhead{Edificios Sustentables}
\chead{Transferencia de calor}
\rhead{Rodrigo Muñoz Sánchez}
\rfoot{\thepage}
\renewcommand{\headrulewidth}{0pt}

\renewcommand{\theenumiii}{\roman{enumiii}}

\title{Clase 25: \\ Reactores acoplados}
\author{Rodrigo Muñoz}
\date{2022}

\graphicspath{Images/}

\begin{document}

\maketitle

\section{Reactores en serie}

Tenemos un sistema con dos reactores conectados entre sí, como se ve en la figura.

Consideramos que tenemos un sistema en estado estacionario y hacemos un balance de masa en cada reactor de la especie de interés:

\[ \begin{aligned}
    1 & : \quad Q \chi_I = Q \chi_1 + k V_1 \chi_1 \\
    2 & : \quad Q \chi_1 = Q \chi_2 + k V_2 \chi_O 
\end{aligned} \]

Reordenamos y despejamos la concentración de salida, utilizando la definición del tiempo de retención:

\[ \chi_I = \chi_1 + k \theta_1 \chi_1 \]

\[ \chi_1 = \frac{ \chi_I }{ 1 + k \theta_1 } \]

\[ \chi_O = \frac{ \chi_1 }{ 1 + k \theta_2 } \]

y llegamos a la ecuación de la concentración para dos estanques acoplados:

\[ \boxed{ \chi_O = \chi_I \left( 1 + k \theta_1 \right) ^ {-1} \left( 1 + k \theta_2 \right) ^ {-1} } \]

Si los dos tiempos de retención son iguales:

\[ \boxed{ \chi_O = \frac{ \chi_I }{ \left( 1 + k \theta \right) ^ 2 } } \]

Si repetimos el proceso podemos llegar al caso general para \( n \) reactores conectados en serie.

Con \( \theta_i \) diferentes:

\[ \boxed{ \chi_O = \chi_I \displaystyle\prod_{i}^{n} \left( 1 + k \theta_i \right) ^ {-1} } \]

Con el mismo tiempo de retención en todos los estanques:

\[ \boxed{ \chi_O = \frac{ \chi_I }{ \left( 1 + k \theta \right) ^ n } } \]

La eficiencia de tratamiento será:

\[ \eta = 1 - \frac{ \chi_O }{ \chi_I } \]

\[ \boxed{ \begin{aligned}
    \eta & = 1 - \prod \left( 1 + k \theta_i \right) ^ {-1} \\
    \eta & = 1 - \left( 1 + k \theta \right) ^ {-n}
\end{aligned} } \]

Comparemos ahora la eficiencia de tratamiento de un reactor con tiempo de retención \( 2 \theta \) contra dos reactores en serie, cada uno con tiempo de retención \( \theta \): 

\[ \eta_A = 1 - \frac{1}{ 1 + 2 k \theta } \]

\[ \eta_B = 1 - \frac{1}{ \left( 1 + k \theta \right) ^ 2 } = 1 - \frac{1}{ 1 + 2 k \theta + k ^ 2 \theta ^ 2 } \]

Vemos que:

\[ \eta_B > \eta_A \]

ya que:

\[ 1 - \frac{1}{ \left( 1 + k \theta \right) ^ 2 } = 1 - \frac{1}{ 1 + 2 k \theta + k ^ 2 \theta ^ 2 } > 1 - \frac{1}{ 1 + 2 k \theta } \]

\[ 1 + 2 k \theta < 1 + 2 k \theta + k ^ 2 \theta ^ 2 \]

\[ 0 < k ^ 2 \theta ^ 2 \]

Hemos llegado a la conclusón que es más eficiente tener reactores en serie, que uno solo del mismo volumen.

\section{Reactores acoplados con recirculación}

En ocasiones, los reactores pueden tener recirculación, lo que permite regresar microorganismos que se encuentran en una etapa de crecimiento óptimo a donde la concentración es mayor, y así mantener un régimen microbiológico ideal.

\begin{itemize}
    \item \textbf{fr}: factor de recirculación, \( 0 \geq fr < 1 \)
\end{itemize}

En este caso el balance de masa es: 

\[ \begin{aligned}
    1 & : \ Q \chi_I + fr Q \chi_O = Q \left( fr + 1 \right) \chi_1 + k V_1 \chi_1 \\
    2 & : \ Q \left( fr + 1 \right) \chi_1 = fr Q \chi_O + k V_2 \chi_O
\end{aligned} \]

Reordenamos el sistema de ecuaciones de dos incógnitas utilizando la definición del tiempo de retención:

\[ \boxed{ \begin{aligned}
    1 & : \ \left( fr + 1 + k \theta_1 \right) \chi_1 - fr \chi_O = \chi_I \\
    2 & : \ \left( -fr - 1 \right) \chi_1 + \left( fr + 1 + \theta_2 k \right) \chi_O = 0
\end{aligned} } \]

El sistema lineal se puede resolver por varios métodos, como la eliminación de Gauss.

Para \( n \) reactores tenemos el siguiente balance de masa:

\[ \begin{aligned}
    1 & : \ \chi_I + fr_1 \chi_2 = \left( fr_1+ 1 + k \theta_1 \right) \chi_1 \\
    2 & : \ \left( fr_1 + 1 \right) \chi_1 + fr_2 \chi_3 = \left( fr_1 + fr_2 + 1 + k \theta_2 \right) \chi_2 \\
    3 & : \ \left( fr_2 + 1 \right) \chi_2 + fr_3 \chi_4 = \left( fr_2 + fr_3 + 1 + k \theta_3 \right) \chi_3 \\
    \vdots & \\
    n & : \ \left( fr_{ n - 1 } + 1 \right) \chi_{ n - 1 } = \left( fr_{ n - 1 } + 1 + k \theta_n \right) \chi_n \\
\end{aligned} \]

Y el sistema de ecuaciones queda como:

\[ \boxed{ \begin{aligned}
    1 & : \ \left( fr_1 + 1 + k \theta_1 \right) \chi_1 - fr_2 \chi_2 = \chi_I \\
    i & : \ \left( -fr_{ i - 1 } - 1 \right) \chi_{ i - 1 } + \left( fr_{ i - 1 } + fr_i + 1 + \theta_i k \right) \chi_i = 0, \quad 2 < i < n \\
    n & : \ \left( -fr_{ n - 1 } - 1 \right) \chi_{ n - 1 } + \left( fr_{ n - 1 } + 1 + \theta_n k \right) \chi_n = 0
\end{aligned} } \]

\section{Reactores en paralelo}

El balance de masa en cada reactor es:

\[ \chi_1 = \frac{ \chi_I }{ 1 + k V_1 Q_{1}^{-1} } \]

\[ \chi_2 = \frac{ \chi_I }{ 1 + k V_2 Q_{2}^{-1} } \]

En el punto de mezclado también hacemos un balance de masa:

\[ \chi_O = \frac{ \chi_1 Q_1 + \chi_2 Q_2 }{ Q_1 + Q_2 } \]

Unimos las tres ecuaciones:

\[ \boxed{ \chi_O = \frac{ \chi_I }{ Q_1 + Q_2 } \left( \frac{ Q_1 } { 1 + k V_1 Q_{1}^{-1} } + \frac{ Q_2 } { 1 + k V_2 Q_{2}^{-1} } \right) } \]

Cuando \( Q_1 = Q_2 \) y \( V_1 = V_2 \)

\[ \chi_1 = \chi_2 \]

\[ \chi_O = \chi_1 \]

Si tenemos \( n \) reactores:

\[ \boxed{ \chi_O = \frac{ \chi_I }{ \sum Q_i } \left( \displaystyle\sum \frac{ Q_i }{ 1 + k + V_i Q_{i}^{-1} } \right) } \]

%\nocite{*}
%\printbibliography

\end{document}