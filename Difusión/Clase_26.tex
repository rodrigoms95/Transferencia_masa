\documentclass[11pt]{article}
\usepackage[a4paper,width=160mm,top=25mm,bottom=25mm]{geometry}
\usepackage{graphicx}
\usepackage{fancyhdr}
\usepackage[spanish, mexico]{babel}
\usepackage[utf8]{inputenc}
\usepackage{amsmath}
\usepackage{gensymb}
\usepackage{upgreek}
\usepackage{mathtools}
\usepackage{xfrac}
\usepackage{hyperref}
%\usepackage[
%    backend=biber,
%    style=apa,
%  ]{biblatex}
\usepackage{csquotes}
%\addbibresource{bibresource}

\setlength{\headheight}{15pt}
\pagestyle{fancy}
\fancyhf{}
\lhead{Edificios Sustentables}
\chead{Transferencia de calor}
\rhead{Rodrigo Muñoz Sánchez}
\rfoot{\thepage}
\renewcommand{\headrulewidth}{0pt}

\renewcommand{\theenumiii}{\roman{enumiii}}

\title{Clase 26: \\ Ejercicios de reactores acoplados}
\author{Rodrigo Muñoz}
\date{2022}

\graphicspath{Images/}

\begin{document}

\maketitle

\section{Ejercicio 1}

Para el estanque facultativo de los ejercicios de reactores bien mezclados, ¿en qué porcentaje se disminuye el área requerida si usamos dos estanques con el mismo volumen? ¿Y si usamos tres?

\bigskip \bigskip

La eficiencias de \( n \) reactores en series es:

\[ \eta = 1 - \left( 1 + k \theta \right) ^ {-n} \]

Despejamos \( \theta \):

\[ \left( 1 + k \theta \right) ^ {-n} = 1 - \eta \]

\[ \left( 1 + k \theta \right) ^ n = \left( 1 - \eta \right) ^ {-1} \]

\[ 1 + k \theta = \left( 1 - \eta \right) ^ { - \sfrac{1}{n} } \]

\[ \boxed{ \theta = \frac{ \left( 1 - \eta \right) ^ { - \sfrac{1}{n} } - 1 }{ k } } \]

El porcentaje de diferencia de área requerida entre 1 reactor y \( n \) reactores acoplados es:

\[ p_n = 1 - \frac{A_n}{A_1} \]

Sustituimos la definición del tiempo de retención:

\[ \theta = \frac{V}{Q} = \frac{ z A }{Q} \]

\[ A = \frac{ \theta Q }{z} \]

\[ A_1 = \frac{Q}{ k z } \left( \frac{ \eta }{ 1 - \eta } \right) \]

\[ A_n = \frac{Q}{ k z } = \left[ \frac{1}{ \left( 1 - \eta \right) } ^ { \sfrac{1}{n} } - 1 \right] \]

\[ A_n = \frac{Q}{ k z } = \left[ \frac{ 1 - \left( 1 - \eta \right) ^ { \sfrac{1}{n} } }{ \left( 1 - \eta \right)  ^ { \sfrac{1}{n} } } \right] \]

Sustituimos en la expresión del porcentaje:

\[ p_n = 1 - \left[ \frac{ 1 - \eta }{ \eta } \right] \left[ \frac{ 1 - \left( 1 - \eta \right) ^ { \sfrac{1}{n} } }{ \left( 1 - \eta \right)  ^ { \sfrac{1}{n} } } \right] \]

\[ p_n = 1 - \left[ \frac{ 1 - \left( 1 - \eta \right) ^ { \sfrac{1}{n} } }{ \eta \left( 1 - \eta \right)  ^ { \sfrac{ 1 - n }{n} } } \right] \]

\[ p_n = 1 - \left[ \frac{ \left( 1 - \eta \right) ^ { 1 - \sfrac{1}{n} } - 1 }{ \eta } + \eta \right] \]

\[ \boxed{ p_n = \frac{ 1 - \left( 1 - \eta \right) ^ { 1 - \sfrac{1}{n} } }{ \eta } } \]

Podemos ver que el porcenaje depende únicamente de la eficiencia y de la cantidad de reactores acoplados, por lo que hemos llegado a un resultado general.

En el caso de \( n = 2 \):

\[ p_2 = \frac{ 1 - \sqrt{ 1 - \eta } }{ \eta } = \frac{ 1 - \sqrt{ 0.333 } }{ 0.667 } = 63.4 \% \]


\( n = 3 \):

\[ p_3 = \frac{ 1 - \left( 1 - \eta \right) ^ { \sfrac{2}{3} } }{ \eta } = \frac{ 1 - \left( 0.333 \right) ^ { \sfrac{2}{3} } }{ 0.667 } = 77.9 \% \]

\( n = 4 \):

\[ p_4 = \frac{ 1 - \left( 1 - \eta \right) ^ { \sfrac{3}{4} } }{ \eta } = \frac{ 1 - \left( 0.333 \right) ^ { \sfrac{3}{4} } }{ 0.667 } = 84.2 \% \]

Entre más reactores acoplados, el porcentaje de reducción de área aumenta, pero ese incremento cada vez es menor.

\section{Ejercicio 2}

Si tenemos dos reactores acoplados iguales con \( k = 0.2 \text{día}^{-1}\), \( \eta = 0.667 \), y \( fr = 0.3 \), calcula \( \theta \).

\bigskip \bigskip

Las ecuaciones para dos tanques con recirculación son: 

\[ \left( fr + 1 + k \theta \right) \chi_1 - fr \chi_O = \chi_I  \]

\[ \left( - fr - 1 \right) \chi_1 + \left( fr + 1 + k \theta \right) \chi_O = 0 \]

Despejamos \( \chi_O \):

\[ \chi_1 = \frac{ \chi_I + fr \chi_O }{ fr + 1 + k \theta } \]

\[ - \left( fr + 1 \right) \frac{ \chi_I + fr \chi_O }{ fr + 1 + k \theta } + \left( fr + 1 + k \theta \right) \chi_O = 0 \]

\[ - \left( fr + 1 \right) \chi_I + \left( fr + 1 + k \theta \right) ^ 2 \chi_O - fr \left( fr + 1 \right) \chi_O = 0 \]

\[ \chi_{ O } = \chi_I \frac{ fr + 1 }{ \left( fr + 1 + k \theta \right) ^ 2 - fr \left( fr + 1 \right) } \]

Sustituimos la eficiencia:

\[ \eta = 1 - \frac{ \chi_O }{ \chi_I } \]

\[ \eta = 1 - \frac{ fr + 1 }{ - fr \left( fr + 1 \right) + \left( fr + 1 + k \theta \right) ^ 2 } \]

\[ \frac{ fr + 1 }{ \left( fr + 1 + k \theta \right) ^ 2 - fr \left( fr + 1 \right) } = 1 - \eta \]

\[ \frac{ fr + 1 }{ 1 - \eta } = \left( fr + 1 + k \theta \right) ^ 2 - fr \left( fr + 1 \right) \]

\[ \begin{aligned}
    \left( fr + 1 \right) \left[ \left( 1 - \eta \right) ^ {-1} + fr \right] & = fr ^ 2 + 1 + k ^ 1 + \theta ^ 2 + 2 fr + 2 k \theta + 2 fr k \theta \\
    & = \left( fr + 1 \right) ^ 2 + k ^ 2 + \theta ^ 2 + 2 fr k \theta
\end{aligned} \]

Reordenamos y usamos la fórmula general para ecuaciones cuadráticas.

\[ \left( k ^ 2 \right) \theta ^ 2 + \left( 2 k + 2 k fr \right) \theta + \left\{ \left( fr + 1 \right) ^ 2 - \left( fr + 1 \right) \left[ \left( 1 - \eta \right) ^ {-1} + fr \right] \right\} \]

\[ \theta = \frac{ - 2 k \left( fr + 1 \right) \pm \left\{ 4 k ^ 2 \left( fr + 1 \right) ^ 2 - 4 k ^ 2 \left( fr + 1 \right) \left[ \left( 1 - \eta \right) ^ {-1} + fr \right] \right\} ^ { \sfrac{1}{2} } }{ 2 k ^ 2 } \]

\[ \boxed{ \theta = \frac{ - \left( fr + 1 \right) + \left( fr + 1 \right) ^ { \sfrac{1}{2} } \left[ \left( 1 - \eta \right) ^ {-1} + fr \right] ^ { \sfrac{1}{2} } }{ k } } \]

Si \( fr = 0 \):

\[ \theta = \frac{ - 1 + \left( 1 - \eta \right) ^ { - \sfrac{1}{2} } }{ k } \]

Calculamos el resultado

\[ \theta = \frac{ -1.3 + 1.3 ^ { \sfrac{1}{2} } \left( 0.333 ^ {-1} + 1.3 \right) ^ { \sfrac{1}{2} } }{ 0.2 } \]

\[ \boxed{ \theta = 5.33 \left[ \text{día} ^ {-1} \right] } \]

\section{Ejercicio 3}

Si en dos reactores acoplados en serie, \( \theta_1 = 2 \theta_2 \), ¿el sistema es más eficiente que si \( \theta_1 = \theta_2 \), o si  \( 2 \theta_1 = \theta_2 \)? En los tres casos considera que \( \theta_T \theta_1 + \theta_2 = \text{cte.} \)

\bigskip \bigskip

Para el caso \( a \), en que \( \theta_1 = 2 \theta_2 \):

\[ \theta_1 = \frac{ \theta_T }{2} \]

y la eficiencia del sistema es:

\[ \eta_a = 1 - \left( 1 + k \frac{ \theta_T }{2} \right) ^ {-2} \]

Para el caso \( b \), en que \( \theta_1 = \theta_2 \):

\[ \theta_T = \theta_1 + \theta_2 = \theta_2 + 2 \theta_2 = 3 \theta 2 \]

\[ \theta_1 = \frac{2}{3} \theta_T \]

\[ \theta_2 = \frac{\theta_T}{3} \]

y la eficiencia del sistema es:

\[ \eta_b = 1 - \left( 1 + k \theta_1 \right) ^ {-1} \left( 1 + k \theta_2 \right) ^ {-1} \]

\[ \eta_b = 1 - \left( 1 + \frac{2}{3} k \theta_T \right) ^ {-1} \left( 1 + k \frac{\theta_T}{3} \right) ^ {-1} \]

El segundo termino del lado derecho de la ecuación de la eficiencia en el caso \( a \) es:

\[ 1 + k \theta_T + k ^ 2 \frac{ \theta_{T}^{2} }{4} = x_a \]

En el caso \( b \) es:

\[ 1 + k \theta_T + \frac{2}{9} k ^ 2 \theta_{T}^{2} = x_b \]

y como:

\[ \frac{1}{4} > \frac{2}{9} \]

entonces: 

\[ x_1 > x_2 \]

\[ x_{a}^{-1} < x_{b}^{-1} \]

\[ \eta_a = 1 - x_{a}^{-1} \]

\[ \eta_b = 1 - x_{b}^{-1} \]

\[ \boxed{ \eta_a > \eta_b } \]

Si repetimos para el caso \( c \), llegamos a que \( \eta_c = \eta_b \)

De esta manera, llegamos a la conclusión que un sistema acoplado es más eficiente cuando todos los tiempos de retención son iguales.

%\nocite{*}
%\printbibliography

\end{document}